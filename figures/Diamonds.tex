% Copyright 2021, C. Minz. BSD 3-Clause License.

\ifdrawdiamondlinklabels
	\def\xaxisticksshift{-2mm}
\else
	\def\xaxisticksshift{0mm}
\fi
%### ignore input error
\input{data/PrefFutures/D\dataD biconeLabels}
%###
\begin{tikzpicture}
	\datavisualization[
		scientific axes=clean,
		x axis={length=\xaxislength cm, min value=1, max value=\xaxismax, label={\ifdrawdiamondlinklabels\else number of center elements\fi}, ticks={step=\xaxisticksstep, node style={yshift=\xaxisticksshift}}},
		y axis={length=\yaxislength cm, min value=0, max value=\yaxismax, label={\datalabel}, ticks={step=\yaxisticksstep, style={/pgf/number format/fixed, /pgf/number format/fixed zerofill, /pgf/number format/precision=\yaxisticksprecision}}},
		all axes={grid},
		visualize as line/.list={1, ..., 6},
		legend={right},
		legend entry options/default label in legend mark/.style=label in legend three marks,
		new legend entry={text={\dataparam}, node style={xshift=-2em}},
		new legend entry={text={}},
		new legend entry={text={$N$:~~%
			\dataNaPadding\textcolor{black!100.0}{\dataNa}%
			\dataNbPadding\textcolor{black!066.7}{\dataNb}%
			\dataNcPadding\textcolor{black!033.3}{\dataNc}}, node style={xshift=-2em}},
		new legend entry={text={Fraction / \%:}, node style={xshift=-2em}},
		1={style={  YorkPurple}, label in legend={text={\legendlabelA}}},
		2={style={     YorkRed}, label in legend={text={\legendlabelB}}},
		3={style={  YorkYellow}, label in legend={text={\legendlabelC}}},
		4={style={   YorkGreen}, label in legend={text={\legendlabelD}}},
		5={style={    YorkBlue}, label in legend={text={\legendlabelE}}},
		6={style={YorkDarkBlue}, label in legend={text={\legendlabelF}}}
		];
	\ifdrawdiamondlinklabels
		\begin{scope}[yshift=\xaxisticksshift]
			% Copyright 2021, C. Minz. BSD 3-Clause License.

\def\xaxislinklabelstep{9/(\xaxismax-1)}
\begin{scope}[xshift=-\xaxislinklabelstep*1cm, yshift=-0.5cm]
	\ifnum\xaxisticksstep=1
		\begin{scope}[xshift=\xaxislinklabelstep*1cm]
			% Copyright 2021, C. Minz. BSD 3-Clause License.

\node[event] (eb) at ( 0.0*\CSunit,-1.1*\CSunit ) {};
\node[inner sep=0.2mm] (e1) at ( 0.0*\CSunit, 0.0*\CSunit ) {\phantom{1}};
\node[event] (et) at ( 0.0*\CSunit, 1.1*\CSunit ) {};
\draw[causalrel] (eb) -- (e1.south);
\draw[causalrel] (e1.north) -- (et);

		\end{scope}
		\foreach \x in {2, 3, 4, ..., \xaxismax}{%
			\begin{scope}[xshift=\xaxislinklabelstep*\x cm]
				\node[event] (eb) at ( 0.0*\CSunit,-1.1*\CSunit ) {};
\node[inner sep=0.4mm] (e1) at ( 0.0*\CSunit, 0.0*\CSunit ) {\phantom{1}};
\node[event] (et) at ( 0.0*\CSunit, 1.1*\CSunit ) {};
\draw[causalrel] (eb) -- (e1.south west);
\draw[causalrel] (eb) -- (e1.south east);
\draw[causalrel] (e1.north west) -- (et);
\draw[causalrel] (e1.north east) -- (et);

			\end{scope}
		}%
	\else
		\pgfmathsetmacro\foreachmax{int(\xaxismax/\xaxisticksstep)}
		\foreach \x in {1, 2, 3, ..., \foreachmax}{%
			\begin{scope}[xshift=\xaxisticksstep*\xaxislinklabelstep*\x cm]
				\node[event] (eb) at ( 0.0*\CSunit,-1.1*\CSunit ) {};
\node[inner sep=0.4mm] (e1) at ( 0.0*\CSunit, 0.0*\CSunit ) {\phantom{1}};
\node[event] (et) at ( 0.0*\CSunit, 1.1*\CSunit ) {};
\draw[causalrel] (eb) -- (e1.south west);
\draw[causalrel] (eb) -- (e1.south east);
\draw[causalrel] (e1.north west) -- (et);
\draw[causalrel] (e1.north east) -- (et);

			\end{scope}
		}%
	\fi
\end{scope}

		\end{scope}
	\fi
	\node[below left] (subvolumesfig) at (\xaxislength, \yaxislength) {
		\begin{tikzpicture}[scale=0.2]
			\pgfsetcornersarced{\pgfpointorigin}
			% Copyright 2021, C. Minz. BSD 3-Clause License.

\def\frameoffset{0.002*\shapesize}
\def\framedecrease{0.05*\shapesize}
\foreach \i / \subvolcolor in {0 / YorkOrange, 1 / YorkYellow, 2 / YorkGreen, 3 / YorkBlue, 4 / YorkDarkBlue, 5 / YorkBlack}
	\draw[draw=\subvolcolor!100, fill=\subvolcolor!100, fill opacity=0.5] 
		   ( 0,-0.5*\shapesize+2*\i*\framedecrease ) % bottom
		-- ( 0.5*\shapesize-\i*\framedecrease, \i*\framedecrease-\i*\frameoffset ) % right
		-- ( 0, 0.5*\shapesize-\i*\frameoffset ) % top
		-- (-0.5*\shapesize+\i*\framedecrease, \i*\framedecrease-\i*\frameoffset ) % left
		-- cycle;

		\end{tikzpicture}
	};
	\node[fill=white, fill opacity=0.5, text opacity=1, scale=0.5] at (subvolumesfig) {$\dataDlabel$};
	\clip (0cm, 0cm) rectangle (\xaxislength, \yaxislength);
	\ifnum\dataNcount>2
		\datavisualization[
			scientific axes=clean,
			x axis={length=\xaxislength cm, min value=1, max value=\xaxismax, ticks={none}},
			y axis={length=\yaxislength cm, min value=0, max value=\yaxismax, ticks={none}},
			all axes={grid=none},
			visualize as line/.list={1, ..., 6},
			1={style={  YorkPurple!033.3}},
			2={style={     YorkRed!033.3}},
			3={style={  YorkYellow!033.3}},
			4={style={   YorkGreen!033.3}},
			5={style={    YorkBlue!033.3}},
			6={style={YorkDarkBlue!033.3}}
			]
			data [set=1, headline={x, y}, format=\dataformat, read from file={\datadir/N\dataNc D\dataD\dataname Vol1Srlsum.csv}]
			data [set=2, headline={x, y}, format=\dataformat, read from file={\datadir/N\dataNc D\dataD\dataname Vol2Srlsum.csv}]
			data [set=3, headline={x, y}, format=\dataformat, read from file={\datadir/N\dataNc D\dataD\dataname Vol3Srlsum.csv}]
			data [set=4, headline={x, y}, format=\dataformat, read from file={\datadir/N\dataNc D\dataD\dataname Vol4Srlsum.csv}]
			data [set=5, headline={x, y}, format=\dataformat, read from file={\datadir/N\dataNc D\dataD\dataname Vol5Srlsum.csv}]
			data [set=6, headline={x, y}, format=\dataformat, read from file={\datadir/N\dataNc D\dataD\dataname Vol6Srlsum.csv}];
	\fi
	\ifnum\dataNcount>1
		\datavisualization[
			scientific axes=clean,
			x axis={length=\xaxislength cm, min value=1, max value=\xaxismax, ticks={none}},
			y axis={length=\yaxislength cm, min value=0, max value=\yaxismax, ticks={none}},
			all axes={grid=none},
			visualize as line/.list={1, ..., 6},
			1={style={  YorkPurple!066.7}},
			2={style={     YorkRed!066.7}},
			3={style={  YorkYellow!066.7}},
			4={style={   YorkGreen!066.7}},
			5={style={    YorkBlue!066.7}},
			6={style={YorkDarkBlue!066.7}}
			]
			data [set=1, headline={x, y}, format=\dataformat, read from file={\datadir/N\dataNb D\dataD\dataname Vol1Srlsum.csv}]
			data [set=2, headline={x, y}, format=\dataformat, read from file={\datadir/N\dataNb D\dataD\dataname Vol2Srlsum.csv}]
			data [set=3, headline={x, y}, format=\dataformat, read from file={\datadir/N\dataNb D\dataD\dataname Vol3Srlsum.csv}]
			data [set=4, headline={x, y}, format=\dataformat, read from file={\datadir/N\dataNb D\dataD\dataname Vol4Srlsum.csv}]
			data [set=5, headline={x, y}, format=\dataformat, read from file={\datadir/N\dataNb D\dataD\dataname Vol5Srlsum.csv}]
			data [set=6, headline={x, y}, format=\dataformat, read from file={\datadir/N\dataNb D\dataD\dataname Vol6Srlsum.csv}];
	\fi
	\ifnum\dataNcount>0
		\datavisualization[
			scientific axes=clean,
			x axis={length=\xaxislength cm, min value=1, max value=\xaxismax, ticks={none}},
			y axis={length=\yaxislength cm, min value=0, max value=\yaxismax, ticks={none}},
			all axes={grid=none},
			visualize as line/.list={1, ..., 6},
			1={style={  YorkPurple!100.0}},
			2={style={     YorkRed!100.0}},
			3={style={  YorkYellow!100.0}},
			4={style={   YorkGreen!100.0}},
			5={style={    YorkBlue!100.0}},
			6={style={YorkDarkBlue!100.0}}
			]
			data [set=1, headline={x, y}, format=\dataformat, read from file={\datadir/N\dataNa D\dataD\dataname Vol1Srlsum.csv}]
			data [set=2, headline={x, y}, format=\dataformat, read from file={\datadir/N\dataNa D\dataD\dataname Vol2Srlsum.csv}]
			data [set=3, headline={x, y}, format=\dataformat, read from file={\datadir/N\dataNa D\dataD\dataname Vol3Srlsum.csv}]
			data [set=4, headline={x, y}, format=\dataformat, read from file={\datadir/N\dataNa D\dataD\dataname Vol4Srlsum.csv}]
			data [set=5, headline={x, y}, format=\dataformat, read from file={\datadir/N\dataNa D\dataD\dataname Vol5Srlsum.csv}]
			data [set=6, headline={x, y}, format=\dataformat, read from file={\datadir/N\dataNa D\dataD\dataname Vol6Srlsum.csv}];
	\fi
\end{tikzpicture}
