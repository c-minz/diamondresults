% Copyright 2021, C. Minz. BSD 3-Clause License.

\begin{tikzpicture}
	\def\datasetcount{4}
	\def\relbargroupwidth{0.8}
	\pgfmathsetmacro\relbarwidth{1.0 * \relbargroupwidth / \datasetcount}
	\pgfmathsetmacro\relbargroupoffset{( 1 - \relbargroupwidth ) / 2}
	\pgfmathsetmacro\relbarpadding{( \relbargroupwidth / \datasetcount - \relbarwidth ) / 2}
	\pgfmathsetmacro\xaxisrange{\xaxismax - \xaxismin}
	\pgfmathsetmacro\xbinwidth{\xbinwidth * \xaxislength / 2 / ( \xaxissplitfrom - \xaxismin )}
	\pgfmathsetmacro\barwidth{\relbarwidth * \xbinwidth}
	\pgfmathsetmacro\xticksshift{\xticksshift * \xaxislength / \xaxisrange}
	\pgfkeyssetvalue{/bar colors/1}{YorkBlack}
	\pgfkeyssetvalue{/bar colors/2}{YorkDarkBlue}
	\pgfkeyssetvalue{/bar colors/3}{YorkBlue}
	\pgfkeyssetvalue{/bar colors/4}{YorkGreen}
	\def\splitoffset{1.0}
	\datavisualization[
		scientific axes=clean,
		all axes={grid},
		x axis={length=0.5*\xaxislength cm, min value=\xaxismin, max value=\xaxissplitfrom, grid={step=0.5}, ticks={major={node style={xshift=\xticksshift cm}, at={0,\xaxisticksstep,...,\xaxissplitfrom}}, step=\xaxisticksstep, style={/pgf/number format/fixed, /pgf/number format/fixed zerofill, /pgf/number format/precision=\xaxisticksprecision}}},
		y axis={length=\yaxislength cm, min value=0, max value=\yaxismax, label={\datalabel}, ticks={step=\yaxisticksstep, style={/pgf/number format/fixed, /pgf/number format/fixed zerofill, /pgf/number format/precision=\yaxisticksprecision}}}
		];
	\datavisualization[
		scientific axes=clean,
		x axis={length=0.5*\xaxislength cm, min value=\xaxismin, max value=\xaxissplitfrom, ticks={none}},
		y axis={length=\yaxislength cm, min value=0, max value=\yaxismax, ticks={none}},
		all axes={grid=none},
		visualize as line/.list={1, ..., 4},
		data/format=ybarleft,
		data/headline={x, y}
		]
		data [set=1, read from file={\datadir/N\dataN D\dataD\dataname Spd1.csv}]
		data [set=2, read from file={\datadir/N\dataN D\dataD\dataname Spd2.csv}]
		data [set=3, read from file={\datadir/N\dataN D\dataD\dataname Spd3.csv}]
		data [set=4, read from file={\datadir/N\dataN D\dataD\dataname Spd4.csv}];
	\begin{scope}[xshift=0.5*\xaxislength cm+\splitoffset cm]
		\pgfmathsetmacro\xaxissecondtick{\xaxissplitto+\xaxisticksstep}
		\datavisualization[
			scientific axes=clean,
			all axes={grid},
			x axis={length=0.5*\xaxislength cm, min value=\xaxissplitto, max value=\xaxismax, grid={step=0.5}, ticks={major={node style={xshift=\xticksshift cm}, at={\xaxissplitto,\xaxissecondtick,...,\xaxismax}}, step=\xaxisticksstep, style={/pgf/number format/fixed, /pgf/number format/fixed zerofill, /pgf/number format/precision=\xaxisticksprecision}}},
			y axis={length=\yaxislength cm, min value=0, max value=\yaxismax, label={}, ticks={none}, grid={step=\yaxisticksstep}},
			visualize as line/.list={1, ..., 4},
			legend={right},
			new legend entry={text={\dataparam}, node style={xshift=-2em}},
			new legend entry={text={}},
			new legend entry={text={4D simplices}, visualizer in legend={\drawbarlegendlabel{4}}},
			new legend entry={text={3D simplices}, visualizer in legend={\drawbarlegendlabel{3}}},
			new legend entry={text={2D simplices}, visualizer in legend={\drawbarlegendlabel{2}}},
			new legend entry={text={none}, visualizer in legend={\drawbarlegendlabel{1}}}
			];
		\datavisualization[
			scientific axes=clean,
			x axis={length=0.5*\xaxislength cm, min value=\xaxissplitto, max value=\xaxismax, ticks={none}},
			y axis={length=\yaxislength cm, min value=0, max value=\yaxismax, ticks={none}},
			all axes={grid=none},
			visualize as line/.list={1, ..., 4},
			data/format=ybarright,
			data/headline={x, y}
			]
			data [set=1, read from file={\datadir/N\dataN D\dataD\dataname Spd1.csv}]
			data [set=2, read from file={\datadir/N\dataN D\dataD\dataname Spd2.csv}]
			data [set=3, read from file={\datadir/N\dataN D\dataD\dataname Spd3.csv}]
			data [set=4, read from file={\datadir/N\dataN D\dataD\dataname Spd4.csv}];
		\fill[white] ( -0.1*\splitoffset, 0 ) rectangle ++( -0.8*\splitoffset, \yaxislength );
		\node[below] at ( -0.5*\splitoffset, \yaxislength ) {%
			\begin{tikzpicture}[scale=0.2]
				%### ignore input error
				\input{figures/ShapeD\dataD bicone}
				%###
			\end{tikzpicture}
		};
		\draw[gray, yshift=-0.206 cm] ( -1.0*\splitoffset, 0 ) 
			-- ( -0.65*\splitoffset, 0 ) 
			-- ( -0.55*\splitoffset,-0.4*\splitoffset ) 
			-- ( -0.45*\splitoffset, 0.4*\splitoffset ) 
			-- ( -0.35*\splitoffset, 0 ) 
			-- ( 0, 0 );
		\node[below=0.7cm] at ( -0.5*\splitoffset, 0 ) 
			{causet fraction / \%};
	\end{scope}
\end{tikzpicture}
