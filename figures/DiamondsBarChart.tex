% Copyright 2021, C. Minz. BSD 3-Clause License.

\ifdrawdiamondlinklabels
	\def\xaxisticksshift{-2mm}
\else
	\def\xaxisticksshift{0mm}
\fi
%### ignore input error
\input{\datadir/D\dataD\dataname}
%###
\begin{tikzpicture}
	\def\datasetcount{6}
	\def\relbargroupwidth{0.8}
	\pgfmathsetmacro\relbarwidth{1.0 * \relbargroupwidth / \datasetcount}
	\pgfmathsetmacro\relbargroupoffset{( 1 - \relbargroupwidth ) / 2}
	\pgfmathsetmacro\relbarpadding{( \relbargroupwidth / \datasetcount - \relbarwidth ) / 2}
	\pgfmathsetmacro\xaxisrange{\xaxismax - \xaxismin}
	\pgfmathsetmacro\xbinwidth{\xbinwidth * \xaxislength / \xaxisrange}
	\pgfmathsetmacro\barwidth{\relbarwidth * \xbinwidth}
	\pgfmathsetmacro\xticksshift{\xticksshift * \xaxislength / \xaxisrange}
	\pgfkeyssetvalue{/bar colors/1}{YorkOrange}
	\pgfkeyssetvalue{/bar colors/2}{YorkYellow}
	\pgfkeyssetvalue{/bar colors/3}{YorkGreen}
	\pgfkeyssetvalue{/bar colors/4}{YorkBlue}
	\pgfkeyssetvalue{/bar colors/5}{YorkDarkBlue}
	\pgfkeyssetvalue{/bar colors/6}{YorkBlack}
	\datavisualization[
		scientific axes=clean,
		x axis={length=\xaxislength cm, min value=\xaxismin, max value=\xaxismax, label={\ifdrawdiamondlinklabels\else number of center elements\fi}, ticks={step=\xaxisticksstep, node style={xshift=\xticksshift cm, yshift=\xaxisticksshift}}},
		y axis={length=\yaxislength cm, min value=0, max value=\yaxismax, label={\datalabel}, ticks={step=\yaxisticksstep, style={/pgf/number format/fixed, /pgf/number format/fixed zerofill, /pgf/number format/precision=\yaxisticksprecision}}},
		all axes={grid},
		visualize as line/.list={1, ..., \datasetcount},
		legend={right},
		new legend entry={text={\dataparam}, node style={xshift=-2em, align=left}},
		new legend entry={text={}},
		new legend entry={text={\legendlabel:}, node style={black, xshift=-2em}},
		new legend entry={text={\legendlabelA}, visualizer in legend={\drawbarlegendlabel{1}}},
		new legend entry={text={\legendlabelB}, visualizer in legend={\drawbarlegendlabel{2}}},
		new legend entry={text={\legendlabelC}, visualizer in legend={\drawbarlegendlabel{3}}},
		new legend entry={text={\legendlabelD}, visualizer in legend={\drawbarlegendlabel{4}}},
		new legend entry={text={\legendlabelE}, visualizer in legend={\drawbarlegendlabel{5}}},
		new legend entry={text={\legendlabelF}, visualizer in legend={\drawbarlegendlabel{6}}}
		];
	\ifdrawdiamondlinklabels
		\begin{scope}[xshift=\xticksshift cm, yshift=\xaxisticksshift]
			% Copyright 2021, C. Minz. BSD 3-Clause License.

\def\xaxislinklabelstep{9/(\xaxismax-1)}
\begin{scope}[xshift=-\xaxislinklabelstep*1cm, yshift=-0.5cm]
	\ifnum\xaxisticksstep=1
		\begin{scope}[xshift=\xaxislinklabelstep*1cm]
			% Copyright 2021, C. Minz. BSD 3-Clause License.

\node[event] (eb) at ( 0.0*\CSunit,-1.1*\CSunit ) {};
\node[inner sep=0.2mm] (e1) at ( 0.0*\CSunit, 0.0*\CSunit ) {\phantom{1}};
\node[event] (et) at ( 0.0*\CSunit, 1.1*\CSunit ) {};
\draw[causalrel] (eb) -- (e1.south);
\draw[causalrel] (e1.north) -- (et);

		\end{scope}
		\foreach \x in {2, 3, 4, ..., \xaxismax}{%
			\begin{scope}[xshift=\xaxislinklabelstep*\x cm]
				% Copyright 2021, C. Minz. BSD 3-Clause License.

\node[event] (eb) at ( 0.0*\CSunit,-1.1*\CSunit ) {};
\node[inner sep=0.4mm] (e1) at ( 0.0*\CSunit, 0.0*\CSunit ) {\phantom{1}};
\node[event] (et) at ( 0.0*\CSunit, 1.1*\CSunit ) {};
\draw[causalrel] (eb) -- (e1.south west);
\draw[causalrel] (eb) -- (e1.south east);
\draw[causalrel] (e1.north west) -- (et);
\draw[causalrel] (e1.north east) -- (et);

			\end{scope}
		}%
	\else
		\pgfmathsetmacro\foreachmax{int(\xaxismax/\xaxisticksstep)}
		\foreach \x in {1, 2, 3, ..., \foreachmax}{%
			\begin{scope}[xshift=\xaxisticksstep*\xaxislinklabelstep*\x cm]
				% Copyright 2021, C. Minz. BSD 3-Clause License.

\node[event] (eb) at ( 0.0*\CSunit,-1.1*\CSunit ) {};
\node[inner sep=0.4mm] (e1) at ( 0.0*\CSunit, 0.0*\CSunit ) {\phantom{1}};
\node[event] (et) at ( 0.0*\CSunit, 1.1*\CSunit ) {};
\draw[causalrel] (eb) -- (e1.south west);
\draw[causalrel] (eb) -- (e1.south east);
\draw[causalrel] (e1.north west) -- (et);
\draw[causalrel] (e1.north east) -- (et);

			\end{scope}
		}%
	\fi
\end{scope}

		\end{scope}
	\fi
	\node[below left] (subvolumesfig) at (\xaxislength, \yaxislength) {
		\begin{tikzpicture}[scale=0.2]
			\pgfsetcornersarced{\pgfpointorigin}
			% Copyright 2021, C. Minz. BSD 3-Clause License.

\def\frameoffset{0.002*\shapesize}
\def\framedecrease{0.05*\shapesize}
\foreach \i / \subvolcolor in {0 / YorkOrange, 1 / YorkYellow, 2 / YorkGreen, 3 / YorkBlue, 4 / YorkDarkBlue, 5 / YorkBlack}
	\draw[draw=\subvolcolor!100, fill=\subvolcolor!100, fill opacity=0.5] 
		   ( 0,-0.5*\shapesize+2*\i*\framedecrease ) % bottom
		-- ( 0.5*\shapesize-\i*\framedecrease, \i*\framedecrease-\i*\frameoffset ) % right
		-- ( 0, 0.5*\shapesize-\i*\frameoffset ) % top
		-- (-0.5*\shapesize+\i*\framedecrease, \i*\framedecrease-\i*\frameoffset ) % left
		-- cycle;

		\end{tikzpicture}
	};
	\node[fill=white, fill opacity=0.75, text opacity=1, scale=0.5] at (subvolumesfig) {$\dataDlabel$};
	\clip (0cm, 0cm) rectangle (\xaxislength, \yaxislength);
	\datavisualization[
		scientific axes=clean,
		x axis={length=\xaxislength cm, min value=\xaxismin, max value=\xaxismax, ticks={none}},
		y axis={length=\yaxislength cm, min value=0, max value=\yaxismax, ticks={none}},
		all axes={grid=none},
		visualize as line/.list={1, ..., \datasetcount},
		data/format=ybar,
		data/headline={x, y}
		]
		data [set=1, read from file={\datadir/N\dataN D\dataD\dataname Vol1Srlsum.csv}]
		data [set=2, read from file={\datadir/N\dataN D\dataD\dataname Vol2Srlsum.csv}]
		data [set=3, read from file={\datadir/N\dataN D\dataD\dataname Vol3Srlsum.csv}]
		data [set=4, read from file={\datadir/N\dataN D\dataD\dataname Vol4Srlsum.csv}]
		data [set=5, read from file={\datadir/N\dataN D\dataD\dataname Vol5Srlsum.csv}]
		data [set=6, read from file={\datadir/N\dataN D\dataD\dataname Vol6Srlsum.csv}];
\end{tikzpicture}
