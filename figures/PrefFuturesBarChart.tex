% Copyright 2021, C. Minz. BSD 3-Clause License.

\begin{tikzpicture}
	\def\datasetcount{6}
	\def\relbargroupwidth{0.8}
	\pgfmathsetmacro\relbarwidth{1.0 * \relbargroupwidth / \datasetcount}
	\pgfmathsetmacro\relbargroupoffset{( 1 - \relbargroupwidth ) / 2}
	\pgfmathsetmacro\relbarpadding{( \relbargroupwidth / \datasetcount - \relbarwidth ) / 2}
	\pgfmathsetmacro\xaxisrange{\xaxismax - \xaxismin}
	\pgfmathsetmacro\xbinwidth{\xbinwidth * \xaxislength / \xaxisrange}
	\pgfmathsetmacro\barwidth{\relbarwidth * \xbinwidth}
	\pgfmathsetmacro\xticksshift{\xticksshift * \xaxislength / \xaxisrange}
	\pgfkeyssetvalue{/bar colors/1}{YorkBlack}
	\pgfkeyssetvalue{/bar colors/2}{YorkDarkBlue}
	\pgfkeyssetvalue{/bar colors/3}{YorkBlue}
	\pgfkeyssetvalue{/bar colors/4}{YorkPurple}
	\pgfkeyssetvalue{/bar colors/5}{YorkGreen}
	\pgfkeyssetvalue{/bar colors/6}{YorkOrange}
	\datavisualization[
		scientific axes=clean,
		x axis={length=\xaxislength cm, min value=\xaxismin, max value=\xaxismax, label={\dataxlabel}, ticks={major={node style={xshift=\xticksshift cm}}, step=\xaxisticksstep, style={/pgf/number format/fixed, /pgf/number format/fixed zerofill, /pgf/number format/precision=\xaxisticksprecision}}},
		y axis={length=\yaxislength cm, min value=0, max value=\yaxismax, label={\dataylabel}, ticks={step=\yaxisticksstep, style={/pgf/number format/fixed, /pgf/number format/fixed zerofill, /pgf/number format/precision=\yaxisticksprecision}}},
		all axes={grid},
		visualize as line/.list={1, ..., \datasetcount},
		legend={right},
		new legend entry={text={\dataparam}, node style={black, xshift=-2em}},
		new legend entry={text={}},
		new legend entry={text={pref.\ future:}, node style={black, xshift=-2em}},
		new legend entry={text={crit.\ 1}, visualizer in legend={\drawbarlegendlabel{1}}},
		new legend entry={text={crit.\ 2}, visualizer in legend={\drawbarlegendlabel{2}}},
		new legend entry={text={crit.\ 3}, visualizer in legend={\drawbarlegendlabel{3}}},
		new legend entry={text={crit.\ 4}, visualizer in legend={\drawbarlegendlabel{4}}},
		new legend entry={text={crit.\ 5}, visualizer in legend={\drawbarlegendlabel{5}}},
		new legend entry={text={crit.\ 6}, visualizer in legend={\drawbarlegendlabel{6}}}
		];
	\node[below left] at (\xaxislength, \yaxislength) {%
		\begin{tikzpicture}[scale=0.2]
			%### ignore input error
			\input{figures/D\dataD/ShapeD\dataD bicone}
			%###
		\end{tikzpicture}
	};
	\clip (0cm, 0cm) rectangle (\xaxislength, \yaxislength);
	\datavisualization[
		scientific axes=clean,
		x axis={length=\xaxislength cm, min value=\xaxismin, max value=\xaxismax, ticks={none}},
		y axis={length=\yaxislength cm, min value=0, max value=\yaxismax, ticks={none}},
		all axes={grid=none},
		visualize as line/.list={1, ..., \datasetcount},
		data/format=ybar,
		data/headline={x, y}
		]
		data [set=1, read from file={\datadir/N\dataN D\dataD\datashape Cri1Min1Vol\datavol.csv}]
		data [set=2, read from file={\datadir/N\dataN D\dataD\datashape Cri2Min1Vol\datavol.csv}]
		data [set=3, read from file={\datadir/N\dataN D\dataD\datashape Cri3Min1Vol\datavol.csv}]
		data [set=4, read from file={\datadir/N\dataN D\dataD\datashape Cri4Min1Vol\datavol.csv}]
		data [set=5, read from file={\datadir/N\dataN D\dataD\datashape Cri5Min1Vol\datavol.csv}]
		data [set=6, read from file={\datadir/N\dataN D\dataD\datashape Cri6Min1Vol\datavol.csv}];
\end{tikzpicture}

