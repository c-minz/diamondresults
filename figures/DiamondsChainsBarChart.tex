% Copyright 2021, C. Minz. BSD 3-Clause License.

\ifdrawdiamondlinklabels
	\def\xaxisticksshift{-2mm}
\else
	\def\xaxisticksshift{0mm}
\fi
\begin{tikzpicture}
	\def\datasetcount{4}
	\def\relbargroupwidth{0.8}
	\pgfmathsetmacro\relbarwidth{1.0 * \relbargroupwidth / \datasetcount}
	\pgfmathsetmacro\relbargroupoffset{( 1 - \relbargroupwidth ) / 2}
	\pgfmathsetmacro\relbarpadding{( \relbargroupwidth / \datasetcount - \relbarwidth ) / 2}
	\pgfmathsetmacro\xaxisrange{\xaxismax - \xaxismin}
	\pgfmathsetmacro\xbinwidth{\xbinwidth * \xaxislength / \xaxisrange}
	\pgfmathsetmacro\barwidth{\relbarwidth * \xbinwidth}
	\pgfmathsetmacro\xticksshift{\xticksshift * \xaxislength / \xaxisrange}
	\datavisualization[
		scientific axes=clean,
		x axis={length=\xaxislength cm, min value=\xaxismin, max value=\xaxismax, label={\dataxlabel}, ticks={step=\xaxisticksstep, node style={xshift=\xticksshift cm, yshift=\xaxisticksshift}}},
		y axis={length=\yaxislength cm, min value=0, max value=\yaxismax, label={\datalabel}, ticks={step=\yaxisticksstep, style={/pgf/number format/fixed, /pgf/number format/fixed zerofill, /pgf/number format/precision=\yaxisticksprecision}}},
		all axes={grid},
		visualize as line/.list={1, ..., \datasetcount},
		legend={below},
		new legend entry={text={dimension $d$:}, node style={xshift=-2em, align=left}},
		new legend entry={text={$1 + 1$}, visualizer in legend={\drawbarlegendlabel{1}}, node style={xshift=-1ex}},
		new legend entry={text={$1 + 2$}, visualizer in legend={\drawbarlegendlabel{2}}, node style={xshift=-1ex}},
		new legend entry={text={$1 + 3$}, visualizer in legend={\drawbarlegendlabel{3}}, node style={xshift=-1ex}},
		new legend entry={text={$1 + 4$}, visualizer in legend={\drawbarlegendlabel{4}}, node style={xshift=-1ex}}
		];
	\ifdrawdiamondlinklabels
		\begin{scope}[xshift=\xticksshift cm, yshift=\xaxisticksshift]
			% Copyright 2021, C. Minz. BSD 3-Clause License.

\def\xaxislinklabelstep{9/(\xaxismax-1)}
\begin{scope}[xshift=-\xaxislinklabelstep*1cm, yshift=-0.5cm]
	\ifnum\xaxisticksstep=1
		\begin{scope}[xshift=\xaxislinklabelstep*1cm]
			% Copyright 2021, C. Minz. BSD 3-Clause License.

\node[event] (eb) at ( 0.0*\CSunit,-1.1*\CSunit ) {};
\node[inner sep=0.2mm] (e1) at ( 0.0*\CSunit, 0.0*\CSunit ) {\phantom{1}};
\node[event] (et) at ( 0.0*\CSunit, 1.1*\CSunit ) {};
\draw[causalrel] (eb) -- (e1.south);
\draw[causalrel] (e1.north) -- (et);

		\end{scope}
		\foreach \x in {2, 3, 4, ..., \xaxismax}{%
			\begin{scope}[xshift=\xaxislinklabelstep*\x cm]
				% Copyright 2021, C. Minz. BSD 3-Clause License.

\node[event] (eb) at ( 0.0*\CSunit,-1.1*\CSunit ) {};
\node[inner sep=0.4mm] (e1) at ( 0.0*\CSunit, 0.0*\CSunit ) {\phantom{1}};
\node[event] (et) at ( 0.0*\CSunit, 1.1*\CSunit ) {};
\draw[causalrel] (eb) -- (e1.south west);
\draw[causalrel] (eb) -- (e1.south east);
\draw[causalrel] (e1.north west) -- (et);
\draw[causalrel] (e1.north east) -- (et);

			\end{scope}
		}%
	\else
		\pgfmathsetmacro\foreachmax{int(\xaxismax/\xaxisticksstep)}
		\foreach \x in {1, 2, 3, ..., \foreachmax}{%
			\begin{scope}[xshift=\xaxisticksstep*\xaxislinklabelstep*\x cm]
				% Copyright 2021, C. Minz. BSD 3-Clause License.

\node[event] (eb) at ( 0.0*\CSunit,-1.1*\CSunit ) {};
\node[inner sep=0.4mm] (e1) at ( 0.0*\CSunit, 0.0*\CSunit ) {\phantom{1}};
\node[event] (et) at ( 0.0*\CSunit, 1.1*\CSunit ) {};
\draw[causalrel] (eb) -- (e1.south west);
\draw[causalrel] (eb) -- (e1.south east);
\draw[causalrel] (e1.north west) -- (et);
\draw[causalrel] (e1.north east) -- (et);

			\end{scope}
		}%
	\fi
\end{scope}

		\end{scope}
	\fi
	\ifdrawshapesgeodesiclegend
		% Copyright 2021, C. Minz. BSD 3-Clause License.

\node[below left] (shapeD2geodesic) at (\xaxislength, \yaxislength) {
	\begin{tikzpicture}[scale=0.2]
		\pgfsetcornersarced{\pgfpointorigin}
		% Copyright 2021, C. Minz. BSD 3-Clause License.

% Copyright 2021, C. Minz. BSD 3-Clause License.

\draw[shapeinset] 
		 ( 0,-0.5*\shapesize) coordinate (2DshapeSouth) 
	-- ( 0.5*\shapesize, 0 ) coordinate (2DshapeEast) 
	-- ( 0, 0.5*\shapesize ) coordinate (2DshapeNorth) 
	-- (-0.5*\shapesize, 0 ) coordinate (2DshapeWest) 
	-- cycle;

\clip (-0.5*\shapesize, 0.00*\shapesize ) 
	-- ( 0.0*\shapesize,-0.5*\shapesize ) 
	-- ( 0.5*\shapesize, 0.0*\shapesize ) 
	-- ( 0.0*\shapesize, 0.5*\shapesize ) 
	-- cycle;
\draw[YorkDarkBlue, line width=1pt, decorate, decoration={random steps, segment length=1.5pt, amplitude=0.5pt}] 
     ( 0,-0.5*\shapesize ) 
	.. controls ( 0.2*\shapesize,-0.1*\shapesize ) 
	   and (-0.1*\shapesize, 0.2*\shapesize ) 
	.. ( 0, 0.5*\shapesize );

	\end{tikzpicture}
};
\node[below] (shapeD3geodesic) at (shapeD2geodesic.south) {
	\begin{tikzpicture}[scale=0.2]
		\pgfsetcornersarced{\pgfpointorigin}
		% Copyright 2021, C. Minz. BSD 3-Clause License.

% Copyright 2021, C. Minz. BSD 3-Clause License.

\begin{scope}[shapeinset]
	\draw ( 0, -0.47*\shapesize) coordinate (Past) 
		-- ( 0.50*\shapesize,  0 ) coordinate (Right) 
		-- ( 0,  0.47*\shapesize ) coordinate (Future) 
		-- (-0.50*\shapesize,  0 ) coordinate (Left) 
		-- cycle;
	\draw (Future) 
		-- (-0.28*\shapesize, -0.06*\shapesize ) coordinate (Front) 
		-- (Past) 
		-- ( 0.28*\shapesize,  0.06*\shapesize ) coordinate (Back) 
		-- cycle;
	\draw ( 0, 0 ) circle[x radius=0.497*\shapesize, y radius=0.07*\shapesize];
\end{scope}

\clip (-0.45*\shapesize, 0.00*\shapesize ) 
	-- ( 0.00*\shapesize,-0.45*\shapesize ) 
	-- ( 0.45*\shapesize, 0.00*\shapesize ) 
	-- ( 0.00*\shapesize, 0.45*\shapesize ) 
	-- cycle;
\draw[YorkBlue, line width=1pt, decorate, decoration={random steps, segment length=1.5pt, amplitude=0.5pt}] 
	   ( 0,-0.45*\shapesize ) 
	.. controls ( 0.2*\shapesize,-0.1*\shapesize ) 
	   and (-0.1*\shapesize, 0.2*\shapesize ) 
	.. ( 0, 0.45*\shapesize );

	\end{tikzpicture}
};
\node[below] (shapeD4geodesic) at (shapeD3geodesic.south) {
	\begin{tikzpicture}[scale=0.2]
		\pgfsetcornersarced{\pgfpointorigin}
		% Copyright 2021, C. Minz. BSD 3-Clause License.

% Copyright 2021, C. Minz. BSD 3-Clause License.

\begin{scope}[shapeinset]
	\draw ( 0, -0.47*\shapesize) coordinate (Past) 
		-- ( 0.50*\shapesize,  0 ) coordinate (Right) 
		-- ( 0,  0.47*\shapesize ) coordinate (Future) 
		-- (-0.50*\shapesize,  0 ) coordinate (Left) 
		-- cycle;
	\draw (Future) 
		-- (-0.28*\shapesize, -0.06*\shapesize ) coordinate (Front) 
		-- (Past) 
		-- ( 0.28*\shapesize,  0.06*\shapesize ) coordinate (Back) 
		-- cycle;
	\draw ( 0,  0 ) circle[x radius=0.497*\shapesize, 
												y radius=0.070*\shapesize];
	\draw ( 0,  0 ) circle[x radius=0.280*\shapesize, 
		y radius=0.190*\shapesize, rotate=10];
	\draw[very thin] ( 0,  0 ) circle[x radius=0.497*\shapesize, 
		y radius=0.190*\shapesize];
	\draw (Past) 
		-- ( 0.03*\shapesize , -0.19*\shapesize );
	\draw[very thin] (Past) 
		-- (-0.03*\shapesize ,  0.19*\shapesize );
	\draw (Future) 
		-- (-0.03*\shapesize ,  0.19*\shapesize );
	\draw[very thin] (Future) 
		-- ( 0.03*\shapesize , -0.19*\shapesize );
\end{scope}

\clip (-0.45*\shapesize, 0.00*\shapesize ) 
	-- ( 0.00*\shapesize,-0.45*\shapesize ) 
	-- ( 0.45*\shapesize, 0.00*\shapesize ) 
	-- ( 0.00*\shapesize, 0.45*\shapesize ) 
	-- cycle;
\draw[YorkGreen, line width=1pt, decorate, decoration={random steps, segment length=1.5pt, amplitude=0.5pt}] 
	   ( 0,-0.45*\shapesize ) 
	.. controls ( 0.2*\shapesize,-0.1*\shapesize ) 
	   and (-0.1*\shapesize, 0.2*\shapesize ) 
	.. ( 0, 0.45*\shapesize );

	\end{tikzpicture}
};

	\fi
	\clip (0cm, 0cm) rectangle (\xaxislength, \yaxislength);
	\setDataNLinearlyIncreasing
	\datavisualization[
		scientific axes=clean,
		x axis={length=\xaxislength cm, min value=\xaxismin, max value=\xaxismax, ticks={none}},
		y axis={length=\yaxislength cm, min value=0, max value=\yaxismax, ticks={none}},
		all axes={grid=none},
		visualize as line/.list={1, ..., \datasetcount},
		data/format=ybar,
		data/headline={x, y}
		]
		data [set=1, read from file={\datadir/N\dataNa D2\dataname.csv}]
		data [set=2, read from file={\datadir/N\dataNb D3\dataname.csv}]
		data [set=3, read from file={\datadir/N\dataNc D4\dataname.csv}]
		data [set=4, read from file={\datadir/N\dataNd D5\dataname.csv}]
		data [set=5, read from file={\datadir/N\dataNe D6\dataname.csv}]
		data [set=6, read from file={\datadir/N\dataNf D7\dataname.csv}]
		;
\end{tikzpicture}
