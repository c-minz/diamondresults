\begin{tikzpicture}
	\def\datasetcount{4}
	\def\relbargroupwidth{0.8}
	\pgfmathsetmacro\relbarwidth{1.0 * \relbargroupwidth / \datasetcount}
	\pgfmathsetmacro\relbargroupoffset{( 1 - \relbargroupwidth ) / 2}
	\pgfmathsetmacro\relbarpadding{( \relbargroupwidth / \datasetcount - \relbarwidth ) / 2}
	\pgfmathsetmacro\xaxisrange{\xaxismax - \xaxismin}
	\pgfmathsetmacro\xbinwidth{\xbinwidth * \xaxislength / \xaxisrange}
	\pgfmathsetmacro\barwidth{\relbarwidth * \xbinwidth}
	\pgfmathsetmacro\xticksshift{\xticksshift * \xaxislength / \xaxisrange}
	\pgfkeyssetvalue{/bar colors/1}{YorkBlack}
	\pgfkeyssetvalue{/bar colors/2}{YorkDarkBlue}
	\pgfkeyssetvalue{/bar colors/3}{YorkBlue}
	\pgfkeyssetvalue{/bar colors/4}{YorkGreen}
	\datavisualization[
		scientific axes=clean,
		x axis={length=\xaxislength cm, min value=\xaxismin, max value=\xaxismax, label={causet fraction / \%}, ticks={major={node style={xshift=\xticksshift cm}, at={0,\xaxisticksstep,...,\xaxismax}}, step=\xaxisticksstep, style={/pgf/number format/fixed, /pgf/number format/fixed zerofill, /pgf/number format/precision=\xaxisticksprecision}}},
		y axis={length=\yaxislength cm, min value=0, max value=\yaxismax, label={\datalabel}, ticks={step=\yaxisticksstep, style={/pgf/number format/fixed, /pgf/number format/fixed zerofill, /pgf/number format/precision=\yaxisticksprecision}}},
		all axes={grid},
		visualize as line/.list={1, ..., 4},
		legend={right},
		new legend entry={text={\dataparam}, node style={xshift=-2em}},
		new legend entry={text={}},
		new legend entry={text={\includegraphics[scale=0.6]{figures/CausetDimension/3CausetSimplex2Layers},}, visualizer in legend={\drawbarlegendlabel{4}}},
		new legend entry={text={\includegraphics[scale=0.6]{figures/CausetDimension/3CausetSimplex2LayersFlipped}}},
		new legend entry={text={\includegraphics[scale=0.6]{figures/CausetDimension/2CausetSimplex2Layers}}, visualizer in legend={\drawbarlegendlabel{3}}},
		new legend entry={text={\includegraphics[scale=0.6]{figures/CausetDimension/1CausetSimplex2Layers}, \includegraphics[scale=0.6]{figures/CausetDimension/1CausetSimplex2LayersFlipped}}, visualizer in legend={\drawbarlegendlabel{2}}},
		new legend entry={text={none}, visualizer in legend={\drawbarlegendlabel{1}}}
		];
	\node[below left] at (\xaxislength, \yaxislength) {%
		\begin{tikzpicture}[scale=0.2]
			%### ignore input error
			\input{figures/D\dataD/ShapeD\dataD bicone}
			%###
		\end{tikzpicture}
	};
	\clip (0cm, 0cm) rectangle (\xaxislength, \yaxislength);
	\datavisualization[
		scientific axes=clean,
		x axis={length=\xaxislength cm, min value=\xaxismin, max value=\xaxismax, ticks={none}},
		y axis={length=\yaxislength cm, min value=0, max value=\yaxismax, ticks={none}},
		all axes={grid=none},
		visualize as line/.list={1, ..., 4},
		data/format=ybar,
		data/headline={x, y}
		]
		data [set=1, read from file={\datadir/N\dataN D\dataD\dataname Spd1.csv}]
		data [set=2, read from file={\datadir/N\dataN D\dataD\dataname Spd2.csv}]
		data [set=3, read from file={\datadir/N\dataN D\dataD\dataname Spd3.csv}]
		data [set=4, read from file={\datadir/N\dataN D\dataD\dataname Spd4.csv}];
\end{tikzpicture}
