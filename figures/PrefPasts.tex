\begin{tikzpicture}
	\ifdrawlegend
		\datavisualization[
			scientific axes=clean,
			x axis={length=\xaxislength cm, min value=0, max value=\xaxismax, label={\dataxlabel}, ticks={step=\xaxisticksstep, style={/pgf/number format/fixed, /pgf/number format/fixed zerofill, /pgf/number format/precision=\xaxisticksprecision}}},
			y axis={length=\yaxislength cm, min value=0, max value=\yaxismax, label={\dataylabel}, ticks={step=\yaxisticksstep, style={/pgf/number format/fixed, /pgf/number format/fixed zerofill, /pgf/number format/precision=\yaxisticksprecision}}},
			all axes={grid},
			visualize as smooth line/.list={1, ..., 6},
			style sheet=pref past criteria,
			legend={below, rows=2, right then down},
			new legend entry={text={criterion:}, node style={black, xshift=-1.1cm}},
			1={label in legend={text={\color{black}{1}}}},
			2={label in legend={text={\color{black}{2}}}},
			3={label in legend={text={\color{black}{3}}}},
			new legend entry={text={}},
			4={label in legend={text={\color{black}{4}}}},
			5={label in legend={text={\color{black}{5}}}},
			6={label in legend={text={\color{black}{6}}}}
			];
	\else
		\datavisualization[
			scientific axes=clean,
			x axis={length=\xaxislength cm, min value=0, max value=\xaxismax, label={\dataxlabel}, ticks={step=\xaxisticksstep, style={/pgf/number format/fixed, /pgf/number format/fixed zerofill, /pgf/number format/precision=\xaxisticksprecision}}},
			y axis={length=\yaxislength cm, min value=0, max value=\yaxismax, label={\dataylabel}, ticks={step=\yaxisticksstep, style={/pgf/number format/fixed, /pgf/number format/fixed zerofill, /pgf/number format/precision=\yaxisticksprecision}}},
			all axes={grid},
			visualize as smooth line/.list={1, ..., 6},
			style sheet=pref past criteria
			];
	\fi
	\node[below left] at (\xaxislength, \yaxislength) {%
		\begin{tikzpicture}[scale=0.2]
			%### ignore input error
			\input{figures/ShapeD\dataD bicone}
			%###
		\end{tikzpicture}
	};
	\clip (0cm, 0cm) rectangle (\xaxislength, \yaxislength);
	\ifdrawEvalues
		\datavisualization[
			scientific axes=clean,
			x axis={length=\xaxislength cm, min value=0, max value=\xaxismax, ticks={none}},
			y axis={length=\yaxislength cm, min value=0, max value=\yaxismax, ticks={none}},
			all axes={grid=none},
			visualize as line/.list={1, ..., 6},
			style sheet=pref past criteria,
			1={style={dotted}},
			2={style={dotted}},
			3={style={dotted}},
			4={style={dotted}},
			5={style={dotted}},
			6={style={dotted}}
			]
			data [set=1, headline={x, y}, read from file={\datadir/N\dataN D\dataD\datashape Cri1Min1Vol\datavol ev.csv}]
			data [set=2, headline={x, y}, read from file={\datadir/N\dataN D\dataD\datashape Cri2Min1Vol\datavol ev.csv}]
			data [set=3, headline={x, y}, read from file={\datadir/N\dataN D\dataD\datashape Cri3Min1Vol\datavol ev.csv}]
			data [set=4, headline={x, y}, read from file={\datadir/N\dataN D\dataD\datashape Cri4Min1Vol\datavol ev.csv}]
			data [set=5, headline={x, y}, read from file={\datadir/N\dataN D\dataD\datashape Cri5Min1Vol\datavol ev.csv}]
			data [set=6, headline={x, y}, read from file={\datadir/N\dataN D\dataD\datashape Cri6Min1Vol\datavol ev.csv}];
	\fi
	\datavisualization[
		scientific axes=clean,
		x axis={length=\xaxislength cm, min value=0, max value=\xaxismax, ticks={none}},
		y axis={length=\yaxislength cm, min value=0, max value=\yaxismax, ticks={none}},
		all axes={grid=none},
		visualize as smooth line/.list={1, ..., 6},
		style sheet=pref past criteria,
		1={pin in data={text={crit.\ 1\ifshowpinArrowFirst$\rightarrow$\fi}, when=x is \pinAxpos, pin length=\pinlen}},
		2={pin in data={text={crit.\ 2}, when=x is \pinBxpos, pin length=\pinlen}},
		3={pin in data={text={crit.\ 3}, when=x is \pinCxpos, pin length=\pinlen}},
		4={pin in data={text={crit.\ 4}, when=x is \pinDxpos, pin length=\pinlen}},
		5={pin in data={text={crit.\ 5}, when=x is \pinExpos, pin length=\pinlen}},
		6={pin in data={text={crit.\ 6}, when=x is \pinFxpos, pin length=\pinlen}},
		data/headline={x, y}
		]
		data [set=1, read from file={\datadir/N\dataN D\dataD\datashape Cri1Min1Vol\datavol.csv}]
		data [set=2, read from file={\datadir/N\dataN D\dataD\datashape Cri2Min1Vol\datavol.csv}]
		data [set=3, read from file={\datadir/N\dataN D\dataD\datashape Cri3Min1Vol\datavol.csv}]
		data [set=4, read from file={\datadir/N\dataN D\dataD\datashape Cri4Min1Vol\datavol.csv}]
		data [set=5, read from file={\datadir/N\dataN D\dataD\datashape Cri5Min1Vol\datavol.csv}]
		data [set=6, read from file={\datadir/N\dataN D\dataD\datashape Cri6Min1Vol\datavol.csv}];
\end{tikzpicture}

