% Copyright 2021, C. Minz. BSD 3-Clause License.

\ifdrawdiamondlinklabels
	\def\xaxisticksshift{-2mm}
\else
	\def\xaxisticksshift{0mm}
\fi
\begin{tikzpicture}
	\datavisualization[
		scientific axes=clean,
		x axis={length=\xaxislength cm, min value=1, max value=\xaxismax, label={\ifdrawdiamondlinklabels\else number of center elements\fi}, ticks={step=\xaxisticksstep, node style={yshift=\xaxisticksshift}}},
		y axis={length=\yaxislength cm, min value=\yaxismin, max value=\yaxismax, label={\datalabel}, ticks={step=\yaxisticksstep, style={/pgf/number format/fixed, /pgf/number format/fixed zerofill, /pgf/number format/precision=\yaxisticksprecision}}},
		all axes={grid},
		visualize as line/.list={theoD2, theoD3, theoD4},
		visualize as scatter/.list={D2, D3, D4},
		style sheet=cross marks,
		legend={right},
		new legend entry={text={\textcolor{black}{\dataparam}}, node style={xshift=-2em}},
		new legend entry={text={}},
		new legend entry={text={sprinkling\\dimension $d$:}, node style={xshift=-2em, align=left}},
		theoD2={style={D2Color}, label in legend={text={$1 + 1$}}},
		theoD3={style={D3Color}, label in legend={text={$1 + 2$}}},
		theoD4={style={D4Color}, label in legend={text={$1 + 3$}}},
		];
	\ifdrawdiamondlinklabels
		\begin{scope}[yshift=\xaxisticksshift]
			% Copyright 2021, C. Minz. BSD 3-Clause License.

\def\xaxislinklabelstep{9/(\xaxismax-1)}
\begin{scope}[xshift=-\xaxislinklabelstep*1cm, yshift=-0.5cm]
	\ifnum\xaxisticksstep=1
		\begin{scope}[xshift=\xaxislinklabelstep*1cm]
			% Copyright 2021, C. Minz. BSD 3-Clause License.

\node[event] (eb) at ( 0.0*\CSunit,-1.1*\CSunit ) {};
\node[inner sep=0.2mm] (e1) at ( 0.0*\CSunit, 0.0*\CSunit ) {\phantom{1}};
\node[event] (et) at ( 0.0*\CSunit, 1.1*\CSunit ) {};
\draw[causalrel] (eb) -- (e1.south);
\draw[causalrel] (e1.north) -- (et);

		\end{scope}
		\foreach \x in {2, 3, 4, ..., \xaxismax}{%
			\begin{scope}[xshift=\xaxislinklabelstep*\x cm]
				% Copyright 2021, C. Minz. BSD 3-Clause License.

\node[event] (eb) at ( 0.0*\CSunit,-1.1*\CSunit ) {};
\node[inner sep=0.4mm] (e1) at ( 0.0*\CSunit, 0.0*\CSunit ) {\phantom{1}};
\node[event] (et) at ( 0.0*\CSunit, 1.1*\CSunit ) {};
\draw[causalrel] (eb) -- (e1.south west);
\draw[causalrel] (eb) -- (e1.south east);
\draw[causalrel] (e1.north west) -- (et);
\draw[causalrel] (e1.north east) -- (et);

			\end{scope}
		}%
	\else
		\pgfmathsetmacro\foreachmax{int(\xaxismax/\xaxisticksstep)}
		\foreach \x in {1, 2, 3, ..., \foreachmax}{%
			\begin{scope}[xshift=\xaxisticksstep*\xaxislinklabelstep*\x cm]
				% Copyright 2021, C. Minz. BSD 3-Clause License.

\node[event] (eb) at ( 0.0*\CSunit,-1.1*\CSunit ) {};
\node[inner sep=0.4mm] (e1) at ( 0.0*\CSunit, 0.0*\CSunit ) {\phantom{1}};
\node[event] (et) at ( 0.0*\CSunit, 1.1*\CSunit ) {};
\draw[causalrel] (eb) -- (e1.south west);
\draw[causalrel] (eb) -- (e1.south east);
\draw[causalrel] (e1.north west) -- (et);
\draw[causalrel] (e1.north east) -- (et);

			\end{scope}
		}%
	\fi
\end{scope}

		\end{scope}
	\fi
	\clip (0cm, 0cm) rectangle (\xaxislength, \yaxislength);
	\datavisualization[
		scientific axes=clean,
		x axis={length=\xaxislength cm, min value=1, max value=\xaxismax, ticks={none}},
		y axis={length=\yaxislength cm, min value=\yaxismin, max value=\yaxismax, ticks={none}},
		all axes={grid=none},
		visualize as line/.list={theoD2, theoD3, theoD4},
		theoD2={style={D2Color}},
		theoD3={style={D3Color}},
		theoD4={style={D4Color}},
		data/format=xcutoff,
		]
		data [set=theoD2, read from file={\datadir/D2DiamondTimesN\dataN\dataSource.csv}]
		data [set=theoD3, read from file={\datadir/D3DiamondTimesN\dataN\dataSource.csv}]
		data [set=theoD4, read from file={\datadir/D4DiamondTimesN\dataN\dataSource.csv}];
	\datavisualization[
		scientific axes=clean,
		x axis={length=\xaxislength cm, min value=1, max value=\xaxismax, ticks={none}},
		y axis={length=\yaxislength cm, min value=\yaxismin, max value=\yaxismax, ticks={none}},
		all axes={grid=none},
		visualize as scatter/.list={D2, D3, D4},
		style sheet=cross marks,
		data/format=\dataformat,
		data/headline={x, y}
		]
		data [set=D2, read from file={\datadir/N\dataN D2\dataname.csv}]
		data [set=D3, read from file={\datadir/N\dataN D3\dataname.csv}]
		data [set=D4, read from file={\datadir/N\dataN D4\dataname.csv}];
	\node[below] at (\shapeposA*\xaxislength, \yaxislength) {%
		\begin{tikzpicture}[shapeinset/.style={draw=D2Color, thin}, scale=\shapescaling]
			% Copyright 2021, C. Minz. BSD 3-Clause License.

\draw[shapeinset] 
		 ( 0,-0.5*\shapesize) coordinate (2DshapeSouth) 
	-- ( 0.5*\shapesize, 0 ) coordinate (2DshapeEast) 
	-- ( 0, 0.5*\shapesize ) coordinate (2DshapeNorth) 
	-- (-0.5*\shapesize, 0 ) coordinate (2DshapeWest) 
	-- cycle;

		\end{tikzpicture}
	};
	\node[above left] at (\xaxislength, \shapeposB*\yaxislength) {%
		\begin{tikzpicture}[shapeinset/.style={draw=D3Color, thin}, scale=\shapescaling]
			% Copyright 2021, C. Minz. BSD 3-Clause License.

\begin{scope}[shapeinset]
	\draw ( 0, -0.47*\shapesize) coordinate (Past) 
		-- ( 0.50*\shapesize,  0 ) coordinate (Right) 
		-- ( 0,  0.47*\shapesize ) coordinate (Future) 
		-- (-0.50*\shapesize,  0 ) coordinate (Left) 
		-- cycle;
	\draw (Future) 
		-- (-0.28*\shapesize, -0.06*\shapesize ) coordinate (Front) 
		-- (Past) 
		-- ( 0.28*\shapesize,  0.06*\shapesize ) coordinate (Back) 
		-- cycle;
	\draw ( 0, 0 ) circle[x radius=0.497*\shapesize, y radius=0.07*\shapesize];
\end{scope}

		\end{tikzpicture}
	};
	\node[below left] at (\xaxislength, \shapeposC*\yaxislength) {%
		\begin{tikzpicture}[shapeinset/.style={draw=D4Color, thin}, scale=\shapescaling]
			% Copyright 2021, C. Minz. BSD 3-Clause License.

\begin{scope}[shapeinset]
	\draw ( 0, -0.47*\shapesize) coordinate (Past) 
		-- ( 0.50*\shapesize,  0 ) coordinate (Right) 
		-- ( 0,  0.47*\shapesize ) coordinate (Future) 
		-- (-0.50*\shapesize,  0 ) coordinate (Left) 
		-- cycle;
	\draw (Future) 
		-- (-0.28*\shapesize, -0.06*\shapesize ) coordinate (Front) 
		-- (Past) 
		-- ( 0.28*\shapesize,  0.06*\shapesize ) coordinate (Back) 
		-- cycle;
	\draw ( 0,  0 ) circle[x radius=0.497*\shapesize, 
												y radius=0.070*\shapesize];
	\draw ( 0,  0 ) circle[x radius=0.280*\shapesize, 
		y radius=0.190*\shapesize, rotate=10];
	\draw[very thin] ( 0,  0 ) circle[x radius=0.497*\shapesize, 
		y radius=0.190*\shapesize];
	\draw (Past) 
		-- ( 0.03*\shapesize , -0.19*\shapesize );
	\draw[very thin] (Past) 
		-- (-0.03*\shapesize ,  0.19*\shapesize );
	\draw (Future) 
		-- (-0.03*\shapesize ,  0.19*\shapesize );
	\draw[very thin] (Future) 
		-- ( 0.03*\shapesize , -0.19*\shapesize );
\end{scope}

		\end{tikzpicture}
	};
\end{tikzpicture}
