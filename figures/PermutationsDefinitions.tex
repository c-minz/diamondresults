\newcounter{permui}
\newcounter{permuj}
\newcommand*{\permcauset}[3][]%
{%
\begin{scope}[#1]
	\begin{scope}[rotate=45]
		\begin{scope}[xshift=-#2*\cellsize/2, yshift=-#2*\cellsize/2]
			\def\permliststr{#3}
			\readlist*\permlisti{\permliststr}
			\readlist*\permlistj{\permliststr}
			% define events and draw labels:
			\setcounter{permui}{0}
			\foreachitem \permvi \in \permlisti[]{
				\stepcounter{permui}
				\coordinate (ec\permvi) at ( \permvi*\cellsize-0.5*\cellsize, \value{permui}*\cellsize-0.5*\cellsize );
				\ifdrawpermutation
					\node[totordlabel] at ( \value{permui}*\cellsize-0.5*\cellsize, 0 ) {\thepermui};
					\path[permcell] ($(ec\permvi)+(-0.5*\cellsize,-0.5*\cellsize )$) rectangle +( 1*\cellsize , 1*\cellsize );
					\node[parordlabel] at ( 0, \value{permui}*\cellsize-0.5*\cellsize ) {\permvi};
				\fi
				\ifdrawbiposet
					\node[event] (e\permvi) at (ec\permvi) {};
				\fi
			}
			% draw shape if option set:
			\ifdrawpermutation
				\draw[2Dshape, fill=none] ( 0*\cellsize, 0*\cellsize ) rectangle ( #2*\cellsize, #2*\cellsize );
			\fi
			\ifdrawbiposet
				% draw causal relations:
				\setcounter{permui}{0}
				\foreachitem \permvi \in \permlisti[]{
					\stepcounter{permui}
					\edef\permvjbound{\fpeval{#2 + 1}}
					\setcounter{permuj}{0}
					\foreachitem \permvj \in \permlistj[]{
						\stepcounter{permuj}
						\ifnum\value{permuj}>\value{permui}
							\ifnum\permvi<\permvj
								\ifnum\permvj<\permvjbound
									\draw[causalrel] (e\permvi) -- (e\permvj);
									\edef\permvjbound{\fpeval{\permvj}}
								\fi
							\fi
						\fi
					}
				}
				% draw spacelike relations:
				\setcounter{permui}{0}
				\foreachitem \permvi \in \permlisti[]{
					\stepcounter{permui}
					\edef\permvjbound{0}
					\setcounter{permuj}{0}
					\foreachitem \permvj \in \permlistj[]{
						\stepcounter{permuj}
						\ifnum\value{permuj}>\value{permui}
							\ifnum\permvi>\permvj
								\ifnum\permvj>\permvjbound
									\draw[spacelikerel] (e\permvj) -- (e\permvi);
									\edef\permvjbound{\fpeval{\permvj}}
								\fi
							\fi
						\fi
					}
				}
			\fi
		\end{scope}
	\end{scope}
\end{scope}
}
\tikzset{permframe/.style={black!25!white, thick}}
\newcommand*{\drawframe}[3][none]{\draw[permframe, fill=#1] ( #2*\offset-\offset/2,-\offset/2 ) rectangle +( #3*\offset, \offset );}
\newcommand*{\drawframeOne}[1][none]{\draw[permframe, fill=#1, xshift=-\offset/2, yshift=-\offset/2] (0, 0) rectangle (\offset, \offset);}
\newcommand*{\drawframeOneShort}[2][0]{\draw[permframe, xshift=-\offset/2, yshift=#1*\offset] (0, 0) rectangle (\offset, #2*\offset);}
\newcommand*{\drawframeTwo}[1][none]{\draw[permframe, fill=#1, xshift=-\offset, yshift=-\offset/2] (0, 0) rectangle (2*\offset, \offset);}
\newcommand*{\drawframeThree}[1][none]{\draw[permframe, fill=#1, xshift=-1.5*\offset, yshift=-\offset/2] (0, 0) rectangle (3*\offset, \offset);}
\newcounter{permrow}

