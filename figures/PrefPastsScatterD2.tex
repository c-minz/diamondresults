% Copyright 2021, C. Minz. BSD 3-Clause License.

\begin{tikzpicture}
	\datavisualization[
		scientific axes=clean,
		x axis={length=\xaxislength cm, label={\dataxlabel}},
		y axis={length=\xaxislength cm, label={\dataylabel}},
		all axes={grid, min value=-\xaxismax, max value=\xaxismax, ticks={step=\xaxisticksstep, style={/pgf/number format/fixed, /pgf/number format/fixed zerofill, /pgf/number format/precision=\xaxisticksprecision}}},
		visualize as scatter/.list={1, ..., 6},
% 		legend={right},
% 		new legend entry={text={\dataparam}, node style={black, xshift=-1.1cm}},
% 		new legend entry={text={}},
% 		new legend entry={text={volume:}, node style={black, xshift=-1.1cm}},
% 		1={style={mark=*, mark size=\markersize, mark options=YorkOrange}, label in legend={text=$V_{0}$, label in legend three marks}},
% 		2={style={mark=*, mark size=\markersize, mark options=YorkYellow}, label in legend={text=$V_{1}$, label in legend three marks}},
% 		3={style={mark=*, mark size=\markersize, mark options=YorkGreen}, label in legend={text=$V_{2}$, label in legend three marks}},
% 		4={style={mark=*, mark size=\markersize, mark options=YorkBlue}, label in legend={text=$V_{3}$, label in legend three marks}},
% 		5={style={mark=*, mark size=\markersize, mark options=YorkDarkBlue}, label in legend={text=$V_{4}$, label in legend three marks}},
% 		6={style={mark=*, mark size=\markersize, mark options=YorkBlack}, label in legend={text=$V_{5}$, label in legend three marks}}
		];
	\clip (0cm, 0cm) rectangle (\xaxislength, \xaxislength);
	\datavisualization[
		scientific axes=clean,
		all axes={grid=none, length=\xaxislength cm, min value=-\xaxismax, max value=\xaxismax, ticks={none}},
		visualize as scatter/.list={6, 5, 4, 3, 2, 1},
% 		1={style={mark=*, mark size=\markersize, mark options={YorkOrange, opacity=\markeropacity}}}, 
% 		2={style={mark=*, mark size=\markersize, mark options={YorkYellow, opacity=\markeropacity}}},
		3={style={mark=*, mark size=\markersize, mark options={YorkGreen, opacity=\markeropacity}}}, 
		4={style={mark=*, mark size=\markersize, mark options={YorkBlue, opacity=\markeropacity}}}, 
		5={style={mark=*, mark size=\markersize, mark options={YorkDarkBlue, opacity=\markeropacity}}}, 
		6={style={mark=*, mark size=\markersize, mark options={YorkBlack, opacity=\markeropacity}}}, 
		data/headline={x, y, z}
		]
% 		data [set=1, read from file={\datadir/N\dataN D\dataD\datashape Cri\dataCri Min1Vol1Scatter.csv}] 
% 		data [set=2, read from file={\datadir/N\dataN D\dataD\datashape Cri\dataCri Min1Vol2Scatter.csv}]
		data [set=3, read from file={\datadir/N\dataN D\dataD\datashape Cri\dataCri Min1Vol3Scatter.csv}]
		data [set=4, read from file={\datadir/N\dataN D\dataD\datashape Cri\dataCri Min1Vol4Scatter.csv}]
		data [set=5, read from file={\datadir/N\dataN D\dataD\datashape Cri\dataCri Min1Vol5Scatter.csv}]
		data [set=6, read from file={\datadir/N\dataN D\dataD\datashape Cri\dataCri Min1Vol6Scatter.csv}];
	\node[below left] (subvolumesfig) at (\xaxislength, \xaxislength) {
		\begin{tikzpicture}[scale=0.2]
			\pgfsetcornersarced{\pgfpointorigin}
			% Copyright 2021, C. Minz. BSD 3-Clause License.

\def\frameoffset{0.002*\shapesize}
\def\framedecrease{0.05*\shapesize}
\foreach \i / \subvolcolor in {0 / YorkOrange, 1 / YorkYellow, 2 / YorkGreen, 3 / YorkBlue, 4 / YorkDarkBlue, 5 / YorkBlack}
	\draw[draw=\subvolcolor!100, fill=\subvolcolor!100, fill opacity=0.5] 
		   ( 0,-0.5*\shapesize+2*\i*\framedecrease ) % bottom
		-- ( 0.5*\shapesize-\i*\framedecrease, \i*\framedecrease-\i*\frameoffset ) % right
		-- ( 0, 0.5*\shapesize-\i*\frameoffset ) % top
		-- (-0.5*\shapesize+\i*\framedecrease, \i*\framedecrease-\i*\frameoffset ) % left
		-- cycle;

		\end{tikzpicture}
	};
	\node[fill=white, fill opacity=0.75, text opacity=1, scale=0.5] at (subvolumesfig) {$\dataDlabel$};
\end{tikzpicture}

